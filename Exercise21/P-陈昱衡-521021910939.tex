\documentclass[]{article}
\usepackage[UTF8]{ctex}
\usepackage[a4paper,left=10mm,right=10mm,bottom=10mm,top=10mm]{geometry}
\usepackage{graphicx}
\usepackage{float}
\usepackage{amsmath,amsfonts,amssymb,amsthm}
\usepackage{array,color}
\newcommand{\image}[2]{\begin{figure}[H]\includegraphics[scale = #1]{#2}\end{figure}}
\usepackage{xcolor} % 用于着色
\usepackage{listings} % 用于排版代码
\usepackage{mathabx}
\usepackage{graphicx} 
\lstset{
basicstyle=\ttfamily\small, % 代码字体大小、样式
keywordstyle=\color{blue}, % 关键词颜色
commentstyle=\color{green!40!black}, % 注释颜色
showstringspaces=false, % 字符串中空格显示为下划线
stringstyle=\color{red!80!black}, % 字符串颜色
numbers=left, % 行号位置:左侧
numberstyle=	iny\color{gray}, % 行号字体大小、样式
stepnumber=1, % 行号递增步长
breaklines=true, % 自动折行
backgroundcolor=\color{lightgray!20}, % 代码块背景色
frame=single, % 代码块边框
tabsize=4, % 缩进长度
captionpos=b, % 标题位置:底部
language=C, % 代码语言
}
%opening
\title{计算机科学中的数学基础Exercise21}
\author{陈昱衡 521021910939}
\date{\today}
\begin{document}
\maketitle

\section*{Basics 14}
\image{0.5}{2023-05-11-11-15-49.png}
因为二项式
\begin{align}
    \binom{n}{k} &= \frac{n!}{k!(n-k)!}
\end{align}
所以按照指数生成函数的定义,原式可以展开为
\begin{align}
    \frac{g_n}{n!} &= -2\frac{g_{n-1}}{(n-1)!} + \sum_{k} \frac{g_{n-k}}{(n-k)!}\frac{g_k}{k!}
\end{align}
根据指数生成函数卷积的定义,有
\begin{align}
    G(z) &= -2z\times G(z) + G(z)^{2} + z
\end{align}

解方程得到
\begin{align}
    G(z) &= \frac{1+2z-\sqrt{1+4z^2}}{2}
\end{align}

对比卡特兰数,可以得到,$g_n$可以使用卡特兰数来表示,
\begin{align}
    g_{2n+1} &= 0\\
    g_{2n} &= (-1)^n(2n)!C_{n-1}
\end{align}

\section*{Homework 21}
\image{0.5}{2023-05-11-11-16-06.png}
对于使用10美金和20美金凑出500美金的问题,我们有
\begin{align}
    10i+20j=500\\
\end{align}
由这种形式,可以想到卷积的定义,借用教材7.3节例4的结论,可以得到本题的生成函数为
\begin{align}
    G(z) &= \frac{1}{1-z^{10}}\frac{1}{1-z^{20}}
\end{align}
如果我们令
\begin{align}
    I &= 10i\\
    J &= 10j \\
\end{align}
我们可以得到
\begin{align}
    I + 2J = 50\\
\end{align}
因此再次利用卷积,紧凑的生成函数
\begin{align}
    % \invhat{G}(z) &= \frac{1}{1-z}\frac{1}{1-z^2}
    \check{G}(z) &= \frac{1}{1-z}\frac{1}{1-z^2}
\end{align}

\subsection*{利用部分分式确定所要求的方法数}

对$\check{G}(z)$进行部分分式,过程如下:
\begin{align}
    \frac{1}{1-z}\frac{1}{1-z^2} &= \frac{A}{1-z} + \frac{B}{1-z^2} + \frac{C}{1+z}\\
\end{align}
有
% \begin{align}
    
% \end{align}
解得:
\begin{gather*}
    \left\{
    \begin{aligned}
        A &= \frac{1}{2}\\
        B &= \frac{1}{4}\\
        C &= \frac{1}{4}\\
    \end{aligned}
    \right.
\end{gather*}
故$\check{G}(z)$部分分式展开为
\begin{align}
    \check{G}(z) &= \frac{1}{2}\frac{1}{1-z} + \frac{1}{4}\frac{1}{1-z^2} + \frac{1}{4}\frac{1}{1+z}
\end{align}
根据指数生成函数的定义,可以得到
\begin{align}
    [z^n]\check{G}(z) &= \frac{1}{2} + \frac{1}{4} \times (-1)^n + \frac{1}{4}(n+1)\\
    &= \frac{3}{4} + \frac{(-1)^n}{4} + \frac{n}{2}
\end{align}

故令$n=50$,有$$[z^{50}]\check{G}(z) = 26$$
共26种给付方法.\par 

\subsection*{利用与(7.39)类 似的方法确定所要求的方法数.}
\image{0.3}{2023-05-12-15-32-02.png}
根据7.39的方法,我们需要把分母凑成$(1-z^2)$的幂次形式,因此有
\begin{align}
    \check{G}(z) &= \frac{1}{1-z}\frac{1}{1-z^2} \\
    &= \frac{1+z}{(1-z^2)^2}\\
\end{align}
根据$\frac{1}{1-z^2}$的生成函数,有
\begin{align}
    \check{G}(z) &= \frac{1+z}{(1-z^2)^2}\\
    &= (z+1)(1+2z^2+3z^4 + \cdots)
\end{align}
因此,我们可以得到
\begin{align}
    [z^n]\check{G}(z) &= \lfloor \frac{n}{2} \rfloor + 1 \\
\end{align}
将$n=50$带入,有$\lfloor \frac{50}{2} \rfloor + 1 = 26$种给付方法。\par

% $\reflectbox{$\hat{\reflectbox{$X$}}$}$
% $\hat{\phantom{x}}$
% $\widehat{x}$
\end{document}